\documentclass[12pt]{jpmletter}

\newcommand{\section}[1]{\textbf{#1}\\}
\newenvironment{original}{\it}{}
\usepackage[rightmargin=0pt, vskip=0pt]{quoting}


\begin{document}
  \begin{letter}{}

    \opening{Dear Editor,}
 
    Thank you very much for the opportunity to re-submit our
    manuscript ``Effectiveness of UNAIDS targets and HIV
    vaccination across 127 countries''. We are grateful for the
    thoughtfulness of the reviews and the appreciation for the
    timeliness and potential impactfulness of our paper. As detailed
    below, we have revised the manuscript to address the reviewers’
    suggestions and questions. The reviewers’ comments have been most
    helpful in strengthening our manuscript.
 
    Thank you for your consideration. We look forward to hearing from
    you.

    \closing{Sincerely,}

    \newpage

    \section{Editor's Remarks to Author}
    \begin{original}
      This paper provides important insights into the impact of
      a partially protective HIV vaccine of the trajectory of the HIV
      pandemic in various regions of the world. The authors should
      make the minor revisions associated with the comments of
      reviewers \#1 and \#3.
    \end{original}
    
    Thank you for your feedback regarding our manuscript. As detailed
    in our point-by-point response, we have addressed the revisions
    requested.

    \section{Reviewer \#1}
    \begin{original}
      Suitable Quality?: Yes\\
      Sufficient General Interest?: Yes\\
      Conclusions Justified?: Yes\\
      Clearly Written?: No\\
      Procedures Described?: Yes\\
      Supplemental Material Warranted?: Yes

      This is an exciting manuscript that demonstrates the power of a
      partially protective vaccine to have an important impact on
      HIV/AIDS incidence and the number of people living with HIV
      (PLHIV), especially when used in programs of vaccine-linked
      chemotherapy with ART.
    \end{original}
    
    Thank you for your appreciation of our work.

    \begin{original}
      1–4. I found the manuscript challenging to read, in part
      because so much information had to be compressed in the brief
      report template used by PNAS. But also because I did not think
      it was well organized—too often the manuscript lapsed into a
      series of unrelated facts. Possibly some organization with
      subheadings might help, if that's acceptable—similar to the
      Global Burden of Disease capstone papers published in the
      Lancet. For example, it would be been helpful to have brief
      paragraphs to look at the impact of vaccination on each of the
      major regions of the world. Doing so would help address some
      interesting features of the model. For example, for the
      incidence data it was not clear to me why the impact of the
      vaccine is so much greater in African countries than in India,
      also for PLHIV. Dividing by region might also be instructive in
      terms of where we need to consider introducing or rolling out
      the vaccine first.
    \end{original}
    
    Thank you for the helpful suggestion of subheadings to clarify the
    organization of our results: we added subheadings to distinguish
    global, regional, country-level model projections, along with
    uncertainty analysis of vaccine characteristics and sensitivity
    analysis.  We have also added the following sentences in the
    regional projections section:
    \begin{quoting}
      The UNAIDS targets based on antiviral treatment affect the
      number of PLHIV by prolonging survival (which increases PLHIV
      compared to status quo) and by reducing transmission (which
      decreases PLHIV). By contrast, vaccination averts new
      infections, reducing PLHIV without the concomitant improvement
      in survival.  This transmission reduction achieved by
      vaccination is greater in African regions that have a high
      incidence, including Swaziland, Uganda, and South Africa, than
      in Asian countries with lower incidence, such as India (Figure
      2).
    \end{quoting}
    
    \begin{original}
      5. It was not clear how long it would take to vaccinate
      a fully immunized individual, and the optimal age group. It
      would have helped to have a brief summary of the vaccine's
      target product profile (TPP).
    \end{original}
    
    Following this suggestion as well as the details provided for
    inclusion by Reviewer \#3, we added a description of the target
    product profile from the HVTN 100/702 trials and expanded the
    description of the most recent vaccine candidate in the context of
    its predecessor:
    \begin{quoting}
      For example, the phase III RV144 trial in Thailand of an
      ALVAC-HIV/gp120 regimen demonstrated efficacies of 60.5\% after
      1 year and of 31.2\% after 3.5 years among participants aged 18
      to 30 years, with vaccine doses administered at weeks 0, 4, 12,
      and 24 (XX). Results of the HVTN 100 trial in South Africa,
      which used an updated ALVAC-HIV/gp120 regimen, showed a more
      robust immune response than RV144 in participants aged 18 to 40
      (XX).  Given the safety and immunogenicity demonstrated by HVTN
      100, the HVTN 702 study was initiated in November 2016 to
      determine HIV vaccine efficacy in 14 sites across South Africa
      (XX). The trial is enrolling 18 to 35 year olds with vaccination
      scheduled at months 0, 1, 3, 6, and 12.  Other advances in
      research on HIV vaccines have included candidates based on
      adenovirus vectors (XX).  Motivated by strong pre-clinical
      scientific underpinnings, three phase I/II trials have been
      initiated as part of the Ad26/MVA/trimer gp140 program and an
      efficacy study is planned for later in 2017 (XX).  In addition,
      investigation into the prophylactic potential of broadly acting
      neutralizing antibodies has provided preliminary evidence for
      enhanced viral clearance through immunotherapy (XX).
    \end{quoting}
    
    \begin{original}
      6. Finally I thought it might be helpful to briefly discuss the
      implications of other partially protective vaccines. More and
      more we're seeing new vaccines come online, which do not have
      the high levels of protection we associate with say measles
      vaccine. For example, in addition to HIV, there's the next
      generation TB, malaria, and other parasite vaccines—increasingly
      such vaccines will not replace existing drug therapies but
      instead will be used alongside them—sometimes called
      ``vaccine-linked chemotherapy''. The analysis here emphasizes
      why and how vaccine-linked chemotherapy can be successfully
      modeled and applied and that should be mentioned.
    \end{original}
    
    Thank you for this excellent suggestion. In the second paragraph
    of the introduction, we have added:
    \begin{quoting}
      Likewise, at the frontier of vaccinology for myriad other
      diseases, including malaria, TB, hookworm, and dengue, is the
      development of partially efficacious vaccines (XX). In cases
      where vaccines may not alone be sufficient for disease
      elimination, they may nonetheless be a component of disease
      control in combination with drug therapies. Here we evaluate
      such vaccine-linked chemotherapy approaches (XX) for HIV.
    \end{quoting}
    
    In the discussion we have also added:
    \begin{quoting}
      Our findings and modeling approaches may have conceptual
      applicability beyond HIV by showing the added benefits of
      partially protective vaccines for other diseases when
      implemented in conjunction with drug therapies. In general,
      rather than expecting vaccines with perfect efficacy to replace
      existing drug therapies, a multifaceted strategy that includes
      partially protective vaccines may nonetheless be necessary to
      achieve elimination, as we demonstrated for HIV.
    \end{quoting}
    
    \section{Reviewer \#2}
    \begin{original}
      Suitable Quality?: Yes \\
      Sufficient General Interest?: Yes \\
      Conclusions Justified?: Yes \\
      Clearly Written?: Yes \\
      Procedures Described?: Yes \\
      Supplemental Material Warranted?: Yes
 
      As I am not a mathematical modeler, my thoughts on this article
      relate to its overall message and presentation to an audience
      interested in HIV prevention and vaccine effectiveness in
      general.

      The bottom line is I like this article for it illustrates some
      interesting issues about a partially effective vaccine, its
      effect on incidence and how it can work synergistically with HIV
      treatment programs such as test and treat. This
      fact/illustration is the most provocative part of the paper and
      in my opinion could be highlighted a bit more, by making the US
      graphs a bit more present in the paper. For example, Figure
      S11—elevate that to a panel to Figure 3, perhaps with Eastern
      and Southern Africa (Figure S7) to show the similarities in
      shape/concept. The paper is well written. The importance of the
      concept is high.
    \end{original}
    
    Thank you for your comments. We agree that the impact of a
    partially effective vaccine alone or in combination with treatment
    is a provocative contribution to the discussion around HIV vaccine
    development. Elevating supplementary figures to the main text
    would further highlight the effects across different settings, yet
    we are concerned that the addition of panels would jeopardize the
    discernibility of Figure 3. Instead, to draw attention to the
    regional trends in the US graphs, we now include references to
    Figures S8 \& S11 in the regional projections section. In the
    Discussion, we have also added the following text to underscore
    the synergy of HIV vaccination with other complementary
    approaches:
    \begin{quoting}
      Our findings and modeling approaches may have conceptual
      applicability beyond HIV by showing the added benefits of
      partially protective vaccines for other diseases when
      implemented in conjunction with drug therapies. In general,
      rather than expecting vaccines with perfect efficacy to replace
      existing drug therapies, a multifaceted strategy that includes
      partially protective vaccines may nonetheless be necessary to
      achieve elimination, as we demonstrated for HIV.
    \end{quoting}
    
    \section{Reviewer \#3}
    \begin{original}
      Suitable Quality?: Yes\\
      Sufficient General Interest?: Yes\\
      Conclusions Justified?: Yes\\
      Clearly Written?: Yes\\
      Procedures Described?: Yes\\
      Supplemental Material Warranted?: Yes 

      Medlock and colleagues describe an infectious disease modeling
      approach to evaluating the impact of HIV vaccines of varying
      degrees of efficacy and global coverage to the global HIV
      pandemic. The paper is concise and well written. The conclusions
      are within the scope of stated limitations of the approach. This
      paper will be of much interest to the field and will be cited by
      many in the HIV vaccine field. Coming at a time of growing
      HIV/AIDS complacency, it's timing couldn't be better. Attention
      to the points below would strengthen the manuscript.

      1. The authors should add a sentence or two describing other HIV
      prevention modalities, such as adult medical male circumcision,
      PrEP, antibody mediated protection, and treatment as prevention,
      that will likely join an effective HIV vaccine as tools to
      mitigate HIV transmission.
    \end{original}

    We have added the following to the discussion section:
    \begin{quoting}
      The vaccination and treatment as prevention approaches
      considered here would be complementary to other HIV prevention
      modalities, such as adult medical male circumcision (XX),
      pre-exposure prophylaxis (PrEP, XX) and antibody-mediated
      protection (XX). The combination of these interventions is
      likely to have synergistic effects on mitigating HIV
      transmission.
    \end{quoting}
    
    \begin{original}
      2. While HVTN 100 was an important study, and the final safety
      and immunogenicity study of ALVAC vCP2348 + gp120 bivalent C
      that led to the decision to launch the HVTN 702 HIV vaccine
      efficacy study, it is the latter study that is more relevant to
      the authors' thesis. HVTN 702 started in mid-November 2016; it
      was highlighted in many 01 December 2016 World AIDS Day
      announcements. This is the first HIV efficacy study to launch in
      seven years and builds on the success of the only HIV vaccine
      efficacy study that showed efficacy, RV 144, which the authors
      cite, but they need to connect HVTN 702 to RV 144 and HVTN 100
      to complete the historical and scientific context.
    \end{original}
    
    Thank you for this important context.  We have provided more
    context to connect the RV 144, HVTN 100, and the HVTN 702 trials,
    as suggested, in the introduction:
    \begin{quoting}
      For example, the phase III RV144 trial in Thailand of an
      ALVAC-HIV/gp120 regimen demonstrated efficacies of 60.5\% after
      1 year and of 31.2\% after 3.5 years among participants aged 18
      to 30 years, with vaccine doses administered at weeks 0, 4, 12,
      and 24 (XX). Results of the HVTN 100 trial in South Africa,
      which used an updated ALVAC-HIV/gp120 regimen, showed a more
      robust immune response than RV144 in participants aged 18 to 40
      (XX).  Given the safety and immunogenicity demonstrated by HVTN
      100, the HVTN 702 study was initiated in November 2016 to
      determine HIV vaccine efficacy in 14 sites across South Africa
      (XX). The trial is enrolling 18 to 35 year olds with vaccination
      scheduled at months 0, 1, 3, 6, and 12.
    \end{quoting}

    \begin{original}
      3. While HIV vaccine development based on ALVAC + gp120 products
      are mentioned in the manuscript, the other major effort in the
      field, the Ad26/MVA/trimer gp140 program is not. This should be
      corrected. The Ad26 program has strong pre-clinical scientific
      underpinnings that have supported advancement into phase I/II
      studies with a first phase IIb efficacy study planned for summer
      2017. While the authors do note that protein immunogens designed
      to elicit broad and potent neutralizing antibody against the
      HIV-1 envelope protein are being explored, they are in much
      earlier clinical development than either the ALVAC + gp120 or
      Ad26 programs.
    \end{original}
    
    We have revised the introduction to reference both efforts
    involving broadly neutralizing antibodies and contributions to
    date of the Ad26/MVA/trimer gp140 program:
    \begin{quoting}
      Other advances in HIV vaccine research have included candidates
      based on adenovirus vectors (XX). Motivated by strong
      pre-clinical scientific underpinnings (XX), three phase I/II
      trials have been initiated as part of the Ad26/MVA/trimer gp140
      program and an efficacy study is planned for later in 2017
      (XX). In addition, investigation into the prophylactic potential
      of broadly acting neutralizing antibodies has provided
      preliminary evidence for enhanced viral clearance through
      immunotherapy (XX).
    \end{quoting}

    \begin{original}
      4. Lines 54–55: International interest in HIV vaccines is
      codified by national HIV vaccine plans published by the
      governments of a number of countries (U.S., Canada, Thailand,
      Nigeria, etc). The fact that national governments have embraced
      HIV vaccine as part of national efforts to control HIV/AIDS is
      significant as it reveals that governments are reacting to the
      data in papers that the authors have cited such as refs 18 \&
      19. Those governments will read this paper with much interest.
    \end{original}
    
    Thank you for your comments supporting the usefulness of our
    manuscript. We are indeed hopeful that our study can contribute to
    support the efforts for HIV vaccine development and deployment. We
    added following in our significance statement:
    \begin{quoting}
      Upon the advent of international partnerships to conduct HIV
      vaccine trials and in anticipation of including vaccination in
      HIV/AIDS control programs, our study provides country-specific
      impacts of a partially effective HIV vaccine and demonstrates
      its importance to the elimination of HIV transmission globally.
    \end{quoting}
    
  \end{letter}

\end{document}
