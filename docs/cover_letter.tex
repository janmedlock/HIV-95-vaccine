\documentclass[12pt]{jpmletter}

\begin{document}
  \begin{letter}{}

    \opening{Dear Editor,}
 
    Please find enclosed our manuscript ``Effectiveness of UNAIDS
    targets and HIV vaccination across 127 countries,'' which we are
    submitting to \textit{PNAS}.
 
    An estimated 37 million people are currently living with HIV. To
    promote higher rates of diagnosis, treatment, and viral
    suppression, the Joint United Nations Programme on HIV/AIDS
    (UNAIDS) has established the 90–90–90 and 95–95–95
    targets. Additionally, recent advances in the development of an
    HIV vaccine have been promising. To evaluate the effectiveness of
    the UNAIDS goals and HIV vaccination, both individually and in
    combination, to turning the tide on the HIV pandemic, we developed
    a mathematical model of HIV progression, transmission, and
    intervention, which we tailored to each of 127 countries that
    collectively harbor over 99\% of HIV infections worldwide.

    This is the first modeling analysis to evaluate the 95–95–95
    UNAIDS target as well as the first broad-scale assessment of the
    global impact of potential HIV vaccination programs. Our results
    show that, relative to maintaining status quo rates of diagnosis
    and treatment, cumulative global HIV incidence in 2035 would be
    54\% and 67\% lower upon meeting the 90–90–90 and 95–95–95
    targets, respectively. The addition of a 50\%-efficacy vaccine in
    2020 gradually scaled up to 70\% coverage, could achieve a further
    26\% reduction by 2035 beyond the 95–95–95 target alone.

    Our results also demonstrate the importance of evaluating
    country-level as well as global intervention effectiveness in the
    prioritization of interventions for country-specific policies. In
    many countries, including the US, Uganda, and Nigeria, the UNAIDS
    goals, as ambitious as they are, would be insufficient to reverse
    the growth of PLHIV without at least a partially efficacious
    vaccine. Even in countries for which UNAIDS goals are sufficient
    to turn the tide on the HIV epidemic, such as India, Tanzania, and
    Ethiopia, vaccination would greatly accelerate elimination, saving
    millions of lives annually.
 
    Thank you for your consideration. We look forward to hearing from
    you.

    \closing{Sincerely,}

  \end{letter}

\end{document}
