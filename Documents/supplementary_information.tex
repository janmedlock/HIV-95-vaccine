\documentclass{article}

\usepackage[utf8]{inputenc}
\usepackage{microtype}
\usepackage{geometry}

\usepackage{amsmath}
\usepackage{graphicx}
\usepackage{tikz}

\usepackage{enumitem}
\setlist{topsep=1ex}

\usepackage[font={sf},
            labelfont={bf},
            labelsep=space,
            singlelinecheck=off]{caption}
\usepackage[font={sf}]{floatrow}
\usepackage{tabularx}

\usepackage[blocks]{authblk}
\renewcommand{\Authsep}{, }
\renewcommand{\Authand}{ \& }
\renewcommand{\Authands}{, }
\renewcommand{\Affilfont}{\small}

\usepackage[pdfborder={0 0 0}]{hyperref}

% \usepackage[style=nature]{biblatex}
% \addbibresource{supplementary_information.bib}
\usepackage[sort&compress]{natbib}
\usepackage{hypernat}
\citestyle{nature}

% For Nature, add 27 to reference numbers to start after
% last number in manuscript.
\usepackage{etoolbox}
\makeatletter
\apptocmd{\thebibliography}{\global\c@NAT@ctr 27\relax}{}{}
\makeatother

% For Nature, no page numbers.
\pagestyle{empty}
\usepackage{etoolbox}
\apptocmd{\maketitle}{\thispagestyle{empty}}{}{}

\usepackage{xr}
\externaldocument{extended_data_fig_1}
\externaldocument{extended_data_fig_2}
\externaldocument{extended_data_fig_3}

\newcommand{\comment}[1]{\textbf{[#1]}}
\newcommand{\md}{\mathrm{d}}
\newcommand{\mT}{\mathrm{T}}
\renewcommand{\vec}[1]{\mathbf{#1}}
\newcommand{\mat}[1]{\mathbf{#1}}
\newcommand{\me}{\mathrm{e}}
\DeclareMathOperator{\Uniform}{Uniform}
\DeclareMathOperator{\Triangular}{Triangular}
\DeclareMathOperator{\Lognormal}{Lognormal}
\DeclareMathOperator{\BetaPERT}{Beta-PERT}

\renewcommand{\thesection}{S\arabic{section}}
\newcommand{\sectionname}{Section}
\renewcommand{\sectionautorefname}{\sectionname}
\renewcommand{\figurename}{Supplementary Fig.}
\renewcommand{\figureautorefname}{\figurename}
\renewcommand{\tablename}{Supplementary Table}
\renewcommand{\tableautorefname}{\tablename}

\title{Supplementary Methods for\\
  \emph{Effectiveness of UNAIDS targets and HIV vaccination across 127
    countries}}

\author{Jan Medlock}
\affil{Department of Biomedical Sciences, Oregon State University, 106
  Dryden Hall, Corvallis, OR, 97331-4801, USA,
  \href{mailto:jan.medlock@oregonstate.edu}
  {\texttt{jan.medlock@oregonstate.edu}}}
\author{Abhishek Pandey} \author{Alyssa S.~Parpia} \author{Amber Tang}
\author{Laura A.~Skrip} \author{Alison P.~Galvani}
\affil{Center for Infectious Disease Modeling and Analysis, Yale
  School of Public Health, 135 College Street, New Haven, USA}


\begin{document}

\maketitle

We developed a country-level mathematical model of HIV transmission
and progression that stratifies HIV infection into acute, chronic
undiagnosed, chronic diagnosed, chronic treated, chronic virally
suppressed, and AIDS, along with uninfected unvaccinated and
vaccinated people (\autoref{model} \&
\hyperref[model_diag]{Extended Data Fig.~\ref*{model_diag}}).
Simulations of our country-level model project the number of people in
each HIV stratum from 2015 to 2035, from which statistics such as
prevalence and incidence were calculated.  Model transition rates were
adjusted dynamically to meet intervention goals for diagnosis,
treatment, viral suppression, and vaccination coverage
(\autoref{targets}).  Data and estimates from many sources were used
to parametrize the model (\autoref{data_sources}).  Country-specific
transmission rates were fitted to historical estimates of the
trajectories of incidence and prevalence, spanning from as early as
1990 for some countries (\autoref{model_fitting}). For every scenario
of intervention combinations, we conducted 1000 model simulations,
sampling values of model parameters from empirical distributions for
each simulation, and summarized the results with median and
percentiles (\autoref{uncertainty}).

The model simulation and analysis tools, written in Python, are
publicly available\cite{medlock2016-git}.


\section{Mathematical model}
\label{model}

We developed a country-level continuous-time mathematical model that
divides people aged $15$\;y and older into non-overlapping HIV states.
We focused on this age group because it accounts for the vast majority
of global PLHIV\cite{UNICEF}, 93\% in 2014; because population-level
HIV estimates and demographic data are typically only estimated for
ages $15$--$49$; and to capture the continuing burden of infections in
people ages $50$ and older.  Our model has 8 HIV states (Extended Data
Fig.~\ref*{model_diag}): susceptible to HIV infection ($S$ is the
number of people susceptible), vaccinated against HIV ($R$), acute HIV
infection ($A$), undiagnosed HIV infection ($U$), diagnosed but
untreated HIV infection ($D$), treated without viral suppression
($T$), viral suppression ($V$), and having AIDS ($W$).  Viral
suppression is defined as a viral load of less than 1000 viral RNA
copies per mL of blood.  Viral suppression not only extends survival
but also dramatically reduces
transmission\cite{vernazza2000, ioannidis2001, attia2009, May2014-gp}.
Simulations of our model project the number of people in each HIV
stratum from 2015 to 2035.

Transitions between model states are governed by a system of
differential equations parametrized using values found in other
studies, along with transmission rates derived from UNAIDS estimates
of incidence and prevalence (\autoref{model_param}).  The model is
deterministic, but uncertainty in the parameters was treated by
running the model 1000 times with samples from the parameter
distributions (\autoref{uncertainty}) and summarizing the model
outcomes using the median and percentiles.  Each country was modeled
separately (Extended Data
Figs.~\ref*{effectiveness_Afghanistan}--\ref*{effectiveness_Zimbabwe})
and the country simulations were aggregated to calculate global and
regional outcomes (Extended Data
Figs.~\ref*{effectiveness_Global}--\ref*{effectiveness_Western_Europe}).

The background mortality rate ($\mu$) in each of the 127 countries was
collected from published
estimates\cite{World_Development_Indicators2013-ee}.  The recruitment
rate ($\kappa$) into the age $15\;\text{y}$ and older group was set by
the difference between the country's population growth
rate\cite{WorldBankpg} and its background mortality
rate\cite{World_Development_Indicators2013-ee}, as would be true if
the population was at its stable age structure.  (More on these data
sources is in \autoref{data_sources}.)

\begin{table}
  \begin{center}
    \begin{tabularx}{\textwidth}{lXlll}
      \hline
      & Definition & Value & Reference \\
      \hline
      $\mu$ & Background death rate
      & Country specific, Supplementary Table 2
      & \cite{World_Development_Indicators2013-ee} \\
      $\kappa$ & Recruitment rate
      & Country specific, Supplementary Table 2
      & \cite{World_Development_Indicators2013-ee, WorldBankpg} \\
      $\delta$	& Rate of leaving acute infection
      & $\Triangular$(2, 4.14, 9.6)\;y$^{-1}$
      & \cite{Hollingsworth2008-iy} \\
      $\sigma$	& Rate of developing AIDS without viral suppression
      & 0.1064\;y$^{-1}$ & \cite{Morgan2002-cq} \\
      $\theta$ & Rate of developing AIDS with viral suppression
      & Eq.~(\ref{theta}) & --- \\
      $\gamma$ & Rate of viral suppression
      & $\Uniform$(0.5, 1.5)\;y$^{-1}$
      & \cite{Currie2009-yz} \\
      $\nu$ & Death rate from AIDS & 0.5\;y$^{-1}$
      & \cite{Morgan2002-cq} \\
      $\rho$ & Vaccine efficacy & 50\%\,[30\%, 70\%] & --- \\
      $\omega$	& Reduction in lifetime with viral suppression
      & $\Uniform$(5, 8)\;y
      & \cite{Samji2013-kf, Unaids2014-ue} \\
      $\tau_{A}$ & Transmissibility during acute phase
      & $\Triangular$(0.0039, 0.0082, 0.0150)
      & \cite{Wawer2005-us, Skarbinski2015-ni}\\
      $\tau_{U}$ & Transmissibility after acute phase
      & $\Triangular$(0.00077, 0.0014, 0.00251)
      & \cite{Hughes2012-so} \\
      $\varepsilon$ & Relative transmissibility with
      viral suppression & $\BetaPERT$(0.08, 0.002, 0.57)
      & \cite{Donnell2010-xo, attia_2009, wilson_2012, jia_2013,
        rodger_2016} \\
      $n$ & Coital acts per year & $\Uniform$(96, 108)
      & \cite{Wawer2005-us, Abdool_Karim2010-cm}\\
      $\alpha$ & Diagnosis rate & See eq.~(\ref{diagnosis_rate}) & --- \\
      $\phi$ & Treatment rate & See eq.~(\ref{treatment_rate}) & --- \\
      $\psi$ & Rate of relapse to untreated & See eq.~(\ref{relapse_rate})
      & --- \\
      $\zeta$ & Vaccination rate & See eq.~(\ref{vaccination_rate}) & --- \\
      \hline
      \vspace*{-2.5em}
    \end{tabularx}
    \caption{Model parameters. $\Triangular$, $\Uniform$, and
      $\BetaPERT$ are sampling distributions (\autoref{uncertainty}).
      For vaccine efficacy, $50\%$ was the baseline, and $30\%$ and
      $70\%$ were also used in a scenario analysis.}
    \label{model_param}
  \end{center}
\end{table}

People susceptible to HIV ($S$) progress to acute infection ($A$)
according to the force of infection ($\lambda$), which depends on the
HIV transmission rate, estimated from the country-specific incidence
and prevalence data (\autoref{model_fitting}), adjusted for the
current number of acutely infected, untreated or ineffectively
treated, and viral suppression PLHIV.  Acute infection is
characterized by high viral loads and lasts an average of 2.90 [95\%
confidence interval 1.23, 6.00]\;months\cite{Hollingsworth2008-iy}.
(We converted these durations to rates like
$1 / 2.90\;\text{month$^{-1}$} \approx 4.14\;\text{y$^{-1}$}$ and set
the parameter using random samples from
$\Triangular(2, 4.14, 9.6)\;\text{y$^{-1}$}$ to capture the
uncertainty in its value.  See \autoref{uncertainty} for the
definition of $\Triangular(a, b, c)$ and other distributions.)  We
assumed that people remain undiagnosed during their acute-infection
phase. After acute infection, people move into the chronic undiagnosed
class ($U$).  People in the undiagnosed class will be diagnosed at the
rate $\alpha$, which is determined by the country's current diagnosis
level and its diagnosis target (\autoref{targets}).  Diagnosed people
($D$) transition to the treated compartment ($T$) upon starting ART at
the rate $\phi$, dependent on the country's current treatment level
and its treatment target (\autoref{targets}).  After some time,
treated people then transition to the viral-suppression class ($V$),
at the rate $\gamma$.  We took $\gamma$ as a random sample from
$\Uniform(0.5, 1.5)\;\text{y$^{-1}$}$, that is, viral suppression
occurs after between 8 months and 2 years on
treatment\cite{Currie2009-yz}.  Disengagement from treatment moves
people from treated ($T$) and viral suppression ($V$) at the rate
$\psi$, dependent on the current level of viral suppression and the
country's target (\autoref{targets}), back to the diagnosed class
($D$).

PLHIV without viral suppression (compartments $U$, $D$, and $T$)
develop AIDS (compartment $W$) after an average duration of 10.4
years\cite{Morgan2002-cq}, i.e. with transition rate
$\sigma = 0.1064\;\text{y$^{-1}$}$.  People who have achieved viral
suppression ($V$) develop AIDS at rate $\theta$, slower than PLHIV
without viral suppression.  People with viral suppression are
estimated to live $5$--$8$\;y shorter lives than the non-HIV
infected\cite{Samji2013-kf, Unaids2014-ue}, which we quantified by
randomly sampling the parameter $\omega$ from
$\Uniform(5, 8)\;\text{y}$.  Assuming no relapse to untreated, the
duration of viral suppression is
\begin{equation}
  \frac{1}{\theta + \mu} = \frac{1}{\mu} - \omega - \frac{1}{\nu}.
\end{equation}
The term to the left of the equals sign is the time until leaving
viral suppression by either progression to AIDS (at rate $\theta$) or
background mortality (at rate $\mu$).  The term to the right of the
equals sign is the lifespan of the non-HIV infected, minus the
reduction in life due to having HIV with viral suppression, minus the
duration of the AIDS stage.  Solving for $\theta$ gives the rate of
progression to AIDS with viral suppression as
\begin{equation}
  \label{theta}
  \theta = \frac{1}{1/\mu - \omega - 1/\nu} - \mu.
\end{equation}

In scenarios with vaccine, susceptible people are vaccinated at rate
$\zeta$, depending on the country's current vaccination coverage and
its coverage target (\autoref{targets}), transitioning into the
vaccinated compartment ($R$), where the vaccine reduces their force of
infection by the efficacy factor $\rho$.  We took the base-case
efficacy to be $50\%$ and varied it to $30\%$ and $70\%$ in analyzing
vaccination scenarios (\autoref{uncertainty}).  We did not model
specific vaccine dose regime, such that implicit in our efficacy
parameter is the assumption that boosters will be used to maintain
efficacy over time.  Without vaccine, $\zeta = 0$.

We also added differential equations to track the cumulative number of
new infections ($Y$) and AIDS deaths ($Z$) over the model period.

The model equations are
\begin{equation}
  \label{model_eqns}
  \begin{split}
    \frac{\md S}{\md t} &= \kappa N - \lambda S - \zeta S- \mu S,
    \\
    \frac{\md R}{\md t} & = \zeta S - (1 - \rho) \lambda R - \mu R,
    \\
    \frac{\md A}{\md t} &= \lambda S + (1 - \rho) \lambda R - \delta A - \mu A,
    \\
    \frac{\md U}{\md t} &= \delta A - \alpha U - \mu U - \sigma U,
    \\
    \frac{\md D}{\md t} &=  \alpha U + \psi T + \psi V
    - \phi D - \mu D - \sigma D,
    \\
    \frac{\md T}{\md t} &= \phi D - \psi T - \gamma T - \mu T
    - \sigma T,
    \\
    \frac{\md V}{\md t} &= \gamma T - \psi V - \mu V - \theta V,
    \\
    \frac{\md W}{\md t} &= \sigma U + \sigma D + \sigma T + \theta V -
    \nu W,
    \\
    \frac{\md Y}{\md t} &= \lambda \big[S + (1 - \rho) R\big],
    \\
    \frac{\md Z}{\md t} &= \nu W,
  \end{split}
\end{equation}
with force of infection
\begin{equation}
  \label{force_of_infection}
  \lambda =
  \frac{\eta \big[\beta_A A + \beta_U (U + D + T) + \beta_V V\big]}{N},
\end{equation}
and sexually active population size $N = S + R + A + U + D + T + V$.
We assumed that people with AIDS ($W$) are too sick to be sexually
active.

The relative transmission rates of acute, unsuppressed, and suppressed
infected people are
\begin{equation}
  \label{betas}
  \begin{split}
    \beta_A &= 1 - (1 - \tau_A)^n,
    \\
    \beta_U &= 1 - (1 - \tau_U)^n,
    \\
    \beta_V &= 1 - (1 - \varepsilon \tau_U)^n,
  \end{split}
\end{equation}
where the $\tau$'s are the relative transmissibilities of people with
acute ($A$), unsuppressed ($U$ and $T$), and suppressed ($V$)
infections, $n$ is the annual number of coital acts, and $\varepsilon$
is the relative transmissibility with viral suppression.  For
transmissibility during the acute phase, we sampled $\tau_A$ from
$\Triangular(0.0039, 0.0082, 0.0150)$\cite{Wawer2005-us,
  Skarbinski2015-ni},
whereas, for transmissibility for unsuppressed people, we sampled
$\tau_U$ from
$\Triangular(0.00077, 0.0014, 0.00251)$\cite{Hughes2012-so}.  The
transmissibility of people with viral suppression relative to those
without suppression has been estimated at 0.08 (95\% confidence
interval 0.002, 0.57)\cite{Donnell2010-xo}: due to the extremely long
left tail, we chose to sample the relative transmissibility
$\varepsilon$ from $\BetaPERT(0.08, 0.002, 0.57)$ rather than a
$\Triangular$ distribution.  (See \autoref{uncertainty} for a full
description of $\BetaPERT(a, b, c)$.)  The annual number of coital
acts $n$ was sampled from
$\Uniform(96, 108)$\cite{Wawer2005-us, Abdool_Karim2010-cm}.  The
country-specific transmission rate $\eta$ was derived from UNAIDS
estimates of each country's longitudinal HIV prevalence and incidence
(\autoref{model_fitting}).

For year 2015, estimates of the number of people aged 15--49 years,
$M(2015)$; the prevalence, $p_I(2015)$; the proportion diagnosed,
$p_D(2015)$; the proportion treated, $p_T(2015)$; and proportion with
viral suppression, $p_V(2015)$ were used to initialize the model
(\autoref{data_sources}).  Although the initial population used ages
15--49 due to the availability of published estimates, model
projections allowed for analysis of infected people aging beyond age
50.  Due to a lack of published estimates of acute infections, and
since the acute phase is so much shorter than the other stages, we
assumed that initially there were no acute infections.  Due to a lack
of comprehensive estimates of the number of people with AIDS in each
country, we assumed that the proportion of diagnosed, untreated people
($D + W$) who have AIDS ($W$) in year 2015 was
\begin{equation}
  p_A = \frac{\sigma}{\sigma + \nu},
\end{equation}
which is the equilibrium fraction of diagnosed people ($D + W$) who
have AIDS ($W$) in the absence of treatment and background mortality.
The number initially vaccinated was set to 0.  The initial cumulative
number of new infections and AIDS deaths were set to 0.  Thus, the
model initial conditions are
\begin{equation}
  \label{initial_conditions}
  \begin{split}
    S(2015) &= \big(1 - p_I(2015)\big) M(2015), \\
    R(2015) &= 0, \\
    A(2015) &= 0, \\
    U(2015) &= \big(1 - p_D(2015)\big) p_I(2015) M(2015), \\
    D(2015) &= \big(1 - p_A\big) p_D(2015) \big(1 - p_T(2015)\big)
    p_I(2015) M(2015), \\
    T(2015) &= p_D(2015) p_T(2015) \big(1 - p_V(2015)\big)
    p_I(2015) M(2015), \\
    V(2015) &= p_D(2015) p_T(2015) p_V(2015) p_I(2015) M(2015), \\
    W(2015) &= p_A p_D(2015) \big(1 - p_T(2015)\big) p_I(2015) M(2015), \\
    Y(2015) &= 0, \\
    Z(2015) &= 0.
  \end{split}
\end{equation}

The parametrized differential equations were solved numerically from
year 2015 to year 2035 using the LSODA routine\cite{odepack, scipy,
  medlock2016-git} and four effectiveness outcomes were computed from
the solutions.
\begin{description}
\item[New infections] New infections from 2015 to time $t$ are given
  by $Y(t)$.

\item[Per-capita incidence] At each solution time point, the
  per-capita incidence was calculated as
  \begin{equation}
    i(t_j) = \frac{Y(t_j) - Y(t_{j - 1})}{M(t_j) (t_j - t_{j - 1})},
  \end{equation}
  where
  \begin{equation}
    M(t) = S(t) + R(t) + A(t) + U(t) + D(t) + T(t) + V(t) + W(t)
  \end{equation}
  is the total population size at time $t$.  The per-capita incidence
  at the first time point was undefined.

\item[PLHIV] At each solution time point, PLHIV was given by
  \begin{equation}
    A(t) + U(t) + D(t) + T(t) + V(t) + W(t).
  \end{equation}

\item[AIDS-related deaths] AIDS-related deaths from 2015 to time $t$
  are given by $Z(t)$.

\end{description}

The results from the country simulations were aggregated to the
regional and global levels by adding the numbers of people in each
compartment over time and then new infections, per-capita incidence,
PLHIV, and HIV-related deaths were calculated from the aggregates.
(See Supplementary Table 6 for the definitions of UNAIDS regions.)


\section{Target rates}
\label{targets}

The current proportion of PLHIV who know their diagnosis is
\begin{equation}
  p_D(t) = \frac{D(t) + T(t) + V(t) + W(t)}
  {A(t) + U(t) + D(t) + T(t) + V(t) + W(t)},
\end{equation}
and the desired target for the current level of diagnosis is
$p_D^*(t)$.  For the rate of new diagnosis, we used the function
\begin{equation}
  \label{diagnosis_rate}
  \alpha(t) = \alpha_{\max} H\big(p_D^*(t) - p_D(t)\big),
\end{equation}
where $H(x)$ is the Heaviside-like function
\begin{equation}
  H(x) =
  \begin{cases}
    0 & \text{if $x < 0$},
    \\
    x / \chi & \text{if $0 \leq x \leq \chi$},
    \\
    1 & \text{if $x > \chi$},
  \end{cases}
\end{equation}
with $\chi = 0.001$, which rapidly switches from $H = 0$ when $x < 0$
to $H = 1$ when $x > 0$.  The linear segment connecting $H = 0$ and
$H = 1$ was used to avoid difficulties in computing numerical
solutions that occur when using a discontinuous function.  The
function for the rate of new diagnoses \eqref{diagnosis_rate} allows
new diagnoses ($\alpha(t) > 0$) when the current diagnosis level is
below the target ($p_D^*(t) > p_D(t)$), and stops new
diagnoses ($\alpha(t) = 0$) when the current diagnosis level is at or
above target.  We took $\alpha_{\max} = 1\;\text{y$^{-1}$}$.

The proportion of diagnosed people who are treated is
\begin{equation}
  p_T(t) = \frac{T(t) + V(t) + W(t)}{D(t) + T(t) + V(t) + W(t)}.
\end{equation}
Like the rate of new diagnosis, the rate of enrolling new people in
treatment is
\begin{equation}
  \label{treatment_rate}
  \phi(t) = \phi_{\max} H\big(p_T^*(t) - p_T(t)\big),
\end{equation}
with $\phi_{\max} = 10$, allowing new treatment ($\phi(t) > 0$) when
the current treatment level is below the target ($p_T^*(t) > p_T(t)$)
and stopping new treatment ($\phi(t) = 0$) when the current treatment
level is at or above target.

The proportion of people on ART who have achieved viral suppression is
\begin{equation}
  p_V(t) = \frac{V(t)}{T(t) + V(t)}.
\end{equation}
The rate of people relapsing to untreated is
\begin{equation}
  \label{relapse_rate}
  \psi(t) = \psi_{\max} H\big(p_V(t) - p_V^*(t)\big),
\end{equation}
with $\psi_{\max} = 1$, which, unlike diagnosis and treatment rates,
prevents relapses ($\psi(t) = 0$) when the current level of viral
suppression is below the target ($p_V^*(t) > p_V(t)$) and allows
relapses ($\psi(t) > 0$) when the current level of viral suppression
is at or above target.  We chose to vary the relapse rate to achieve
the desired level of viral suppression because achieving viral
suppression seems to depend on the time on treatment (i.e.
$1 / \gamma$), while relapse rates are difficult to estimate.

The proportion of susceptible people vaccinated is
\begin{equation}
  p_R(t) = \frac{R(t)}{S(t) + R(t)}.
\end{equation}
Like diagnosis and treatment, the vaccination rate is
\begin{equation}
  \label{vaccination_rate}
  \zeta(t) = \zeta_{\max} H\big(p_R^*(t) - p_R(t)\big),
\end{equation}
with $\zeta_{\max} = 1$, allowing vaccination ($\zeta(t) > 0$) when
the current vaccine coverage level is below the target
($p_R^*(t) > p_R(t)$) and stopping vaccination ($\zeta(t) = 0$) when
the current vaccine coverage is at or above target.

We modeled 3 different targets for the proportion diagnosed ($p_D$), the
proportion treated ($p_T$), and the proportion with viral suppression ($p_V$).
\begin{description}
\item[Status quo] The status quo target is for the proportions $p_D$,
  $p_T$, and $p_V$ to remain fixed at their initial levels
  (Supplementary Table 5) going forward. That is,
  \begin{equation}
    \label{status_quo_target}
    \begin{split}
      p_D^*(t) &= p_D(2015), \\
      p_T^*(t) &= p_T(2015), \\
      p_V^*(t) &= p_V(2015).
    \end{split}
  \end{equation}

\item[UNAIDS 90--90--90] The UNAIDS 90--90--90 target is for $p_D$,
  $p_T$, and $p_V$ to all be raised to $90\%$ by 2020.  We implemented
  this target as linear increases from the 2015 levels, each up to
  $90\%$ in 2020, and constant at $90\%$ from 2020 to 2035, except if
  an initial level is above $90\%$ then it remains constant from 2015
  to 2035 so that meeting the UNAIDS target does not worsen any aspect
  of the treatment cascade.  That is,
  \begin{equation}
    \label{unaids90_targets}
    \begin{split}
      p_D^*(t) &=
      \begin{cases}
        F\big(t, 2015, p_D(2015), 2020, 0.9\big)
        & \text{if $p_D(2015) < 0.9$},
        \\
        p_D(2015) & \text{if $p_D(2015) \geq 0.9$}.
      \end{cases}
      \\
      p_T^*(t) &=
      \begin{cases}
        F\big(t, 2015, p_T(2015), 2020, 0.9\big)
        & \text{if $p_T(2015) < 0.9$},
        \\
        p_T(2015) & \text{if $p_T(2015) \geq 0.9$}.
      \end{cases}
      \\
      p_V^*(t) &=
      \begin{cases}
        F\big(t, 2015, p_V(2015), 2020, 0.9\big)
        & \text{if $p_V(2015) < 0.9$},
        \\
        p_V(2015) & \text{if $p_V(2015) \geq 0.9$}.
      \end{cases}
    \end{split}
  \end{equation}
  where
  \begin{equation}
    \label{F}
    F(t, t_0, x_0, t_1, x_1) =
    \begin{cases}
      x_0 & \text{if $t < t_0$},
      \\
      x_0 + (x_1 - x_0) \frac{t - t_0}{t_1 - t_0} &
      \text{if $t_0 \leq t < t_1$},
      \\
      x_1 & \text{if $t \geq t_1$},
    \end{cases}
  \end{equation}
  is constant at $x_0$ for $t < t_0$, linearly connects $x_0$ at
  $t_0$ to $x_1$ at $t_1$ for $t_0 \leq t < t_1$, and is constant at
  $x_1$ for $t \geq t_1$.

\item[UNAIDS 95--95--95] The UNAIDS 95--95--95 target is to achieve
  90--90--90 by 2020 and then for $p_D$, $p_T$, and $p_V$ to all be
  raised to $95\%$ by 2030.  We implemented this target as for
  90--90--90 from 2015 to 2020, and then linear increases from the
  2020 levels up to $95\%$ in 2030, again with the exception that if
  an initial level is above $95\%$ then it remains constant from 2015
  to 2035.  That is,
  \begin{equation}
    \label{unaids95_targets}
    \begin{split}
      p_D^*(t) &=
      \begin{cases}
        G\big(t, 2015, p_D(2015), 2020, 0.9, 2030, 0.95\big)
        & \text{if $p_D(2015) < 0.9$},
        \\
        F\big(t, 2020, p_D(2015), 2030, 0.95\big)
        & \text{if $0.9 \leq p_D(2015) < 0.95$},
        \\
        p_D(2015) & \text{if $p_D(2015) \geq 0.95$},
      \end{cases}
      \\
      p_T^*(t) &=
      \begin{cases}
        G\big(t, 2015, p_T(2015), 2020, 0.9, 2030, 0.95\big)
        & \text{if $p_T(2015) < 0.9$},
        \\
        F\big(t, 2020, p_T(2015), 2030, 0.95\big)
        & \text{if $0.9 \leq p_T(2015) < 0.95$},
        \\
        p_T(2015) & \text{if $p_T(2015) \geq 0.95$},
      \end{cases}
      \\
      p_V^*(t) &=
      \begin{cases}
        G\big(t, 2015, p_V(2015), 2020, 0.9, 2030, 0.95\big)
        & \text{if $p_V(2015) < 0.9$},
        \\
        F\big(t, 2020, p_V(2015), 2030, 0.95\big)
        & \text{if $0.9 \leq p_V(2015) < 0.95$},
        \\
        p_V(2015) & \text{if $p_V(2015) \geq 0.95$},
      \end{cases}
    \end{split}
  \end{equation}
  where
  \begin{equation}
    G(t, t_0, x_0, t_1, x_1, t_2, x_2) =
    \begin{cases}
      x_0 & \text{if $t < t_0$},
      \\
      x_0 + (x_1 - x_0) \frac{t - t_0}{t_1 - t_0} &
      \text{if $t_0 \leq t < t_1$},
      \\
      x_1 + (x_2 - x_1) \frac{t - t_1}{t_2 - t_1} &
      \text{if $t_1 \leq t < t_2$},
      \\
      x_2 & \text{if $t \geq t_2$},
    \end{cases}
  \end{equation}
  is constant at $x_0$ for $t < t_0$, linearly connects $x_0$ at $t_0$
  to $x_1$ at $t_1$ for $t_0 \leq t < t_1$, linearly connects $x_1$ at
  $t_1$ to $x_2$ at $t_2$ for $t_1 \leq t < t_2$, and is constant at
  $x_2$ for $t \geq t_2$, and $F$ is as in eq.~\eqref{F}.

\end{description}

For vaccination, we assume that rollout begins in year $t_V = 2020$
(or $t_V = 2025$ in vaccine scenario analysis, \autoref{uncertainty}),
that rollout linearly increases the proportion of the non-infected
population covered from 0\% by $r_V = 25\%$ per year (or $r_V = 10\%$
per year) up to $c_V = 70\%$ (or $c_V = 50\%$ or $c_V = 90\%$) and
staying constant thereafter.  Thus, the vaccination target is
\begin{equation}
  \label{vaccination_target}
  p_R^*(t) =
  \begin{cases}
    0 & \text{if $t < t_V$},
    \\
    r_V (t - t_V) & \text{if $t_V \leq t < T_V$},
    \\
    c_V & \text{if $t \geq T_V$},
  \end{cases}
\end{equation}
where $T_V = t_V + c_V / r_V$ is the time when rollout reaches
the ultimate coverage $c_V$.

Target strategies were one of the three diagnosis and treatment
targets (status quo, 90--90--90, or 95--95--95), combined with either
no vaccination ($p_R^*(t) = 0$) or vaccination at some level of
efficacy, ultimate coverage, start date of rollout, and speed of
rollout.


\section{Data sources}
\label{data_sources}

For 127 countries, we found sufficient published estimates and data to
parametrize our model.  Demographic data, including population growth
rate\cite{WorldBankpg}, death
rate\cite{World_Development_Indicators2013-ee}, and number of people
aged 15--49 years\cite{The_World_Bank2016-fd} were obtained from the
World Bank (Supplementary Table 2). Longitudinal HIV prevalence
(Supplementary Table 3) and incidence (Supplementary Table 4)
estimates for ages 15--49, spanning from as early as 1990 for some
countries to 2015, were primarily obtained from the AIDSinfo database
produced by UNAIDS\cite{Unaids2016-an}. Other sources included UNAIDS
Country Progress Reports\cite{Unaids2016-am} published from 2012 to
2016 and AIDS Data Hub\cite{AIDSdatahub-fg}, which compiles data from
UNAIDS, UNICEF, WHO, and the Asian Development Bank. These sources
were also used to inform the initial conditions for the model: the
number of people diagnosed with HIV, the number on ART, and the number
who have viral suppression or have been retained on treatment for at
least 12 months (Supplementary Table 5). Where estimates or data were
not available from these sources, alternative resources such as
peer-reviewed journal articles and country health ministry reports
were consulted.  See Supplementary Tables 2--5 for full information on
the sources used for each country.  These source tables are also
available in the public source-code repository\cite{medlock2016-git}.


\section{Model fitting}
\label{model_fitting}

Using available historical country-level estimates of prevalence and
incidence (Supplementary Tables 3 and 4), and published estimates of
transmissibility\cite{Wawer2005-us, Donnell2010-xo, Hughes2012-so,
  Skarbinski2015-ni},
we derived an average country-specific transmission rate, which then
informs the acute, unsuppressed, and suppressed transmission rates.
This calibration method provided a recent estimate of country-specific
transmission, while using historical data to smooth out short-term
variations.

For each country, we first derived time-dependent rates of HIV
transmission for each of the available points in the historical
estimates of prevalence and incidence.  The calculation involved
approximating the simplified force of infection
\begin{align}
  \label{foi}
  \begin{split}
    \lambda(t) &= \frac{\eta \left[\beta_{A} A(t)
        + \beta_{U} \big(U(t) + D(t) + T(t)\big) +
        \beta_{V} V(t)\right]}{N(t)}
    \approx  \beta(t) \frac{I(t)}{N(t)},
  \end{split}
\end{align}
where $\beta$ is the estimated time-dependent aggregate transmission
rate under the approximation that there is no difference in
transmission risk among acute, unsuppressed, suppressed, and AIDS
classes, and $I = A + U + D + T + V + W$ is the total number of PLHIV.
Therefore, the per-capita incidence is
\begin{equation}
i(t) = \lambda(t) \frac{S(t)}{N(t)}
\approx \beta(t) \frac{I(t)}{N(t)} \frac{S(t)}{N(t)} =\beta(t) p(t) (1-p(t)),
\end{equation}
where $p(t)$ is the prevalence. Thus, we calculated the transmission
rate at each historical time point using the incidence and prevalence
estimates by
\begin{equation}
  \label{trans_rate}
  \beta(t) \approx \frac{i(t)}{p(t)(1-p(t))}.
\end{equation}

To reflect the uncertainty in $\beta(t)$, we assumed that it follows a
lognormal distribution (\autoref{uncertainty}). For the parameters
$\mu$ and $\sigma^2$ of the lognormal distribution we used the
exponentially weighted mean and variance\cite{holt2004} over time
(with half-life 1\;y) of $\log \beta(t)$ for the year 2015.  That is,
\begin{equation}
  \label{lognormal_params}
  \begin{split}
    \mu &= \frac{\sum_{i = 1}^n  2^{t_i}
      \log \beta(t_i)}
    {\sum_{i = 1}^n 2^{t_i}},
    \\
    \sigma^2 &= \frac{\big(\sum_{i = 1}^n 2^{t_i}\big)
      \left[\sum_{i = 1}^n  2^{t_i}
        \big(\log \beta(t_i) - \mu\big)^2\right]}
    {\big(\sum_{i = 1}^n 2^{t_i}\big)^2
      - \sum_{i = 1}^n 2^{2 t_i}}.
  \end{split}
\end{equation}
A half-life of 1\;y means that the estimate of the transmission rate
from year $i$ is weighted half as much as the estimate from year
$i + 1$, a quarter as much as the estimate from year $i + 2$, and so
on.  This approach captured the recent country-level transmission
behavior while still allowing for the use of less-recent estimates to
smooth short-term variations (\autoref{transmission_rate}).

\begin{figure}
  \centering
  \includegraphics{../Codes/plots/transmission_rate.pdf}
  \caption{Country-specific distributions of transmission rate,
    $\bar{\beta}$, calculated based on UNAIDS historical estimates of
    country prevalence and incidence.  See \autoref{model_fitting}.}
  \label{transmission_rate}
\end{figure}

For each of the 1000 model parameter samples, the country-level
aggregate transmission rate $\bar{\beta}$ was sampled from
$\Lognormal(\mu, \sigma^2)$ with the parameters given by
eqs.~\eqref{lognormal_params}, and samples were also drawn for the
transmissibilities for acute and unsuppressed infected people,
$\tau_A$ and $\tau_U$; the transmission reduction by suppression,
$\varepsilon$; and the annual number of coital acts, $n$
(\autoref{model}).  From the latter samples, the relative transmission
rates for the acute phase ($\beta_A$), for unsuppressed chronic
infections ($\beta_U$), and for chronic infections with viral
suppression ($\beta_V$) were calculated by eqs.~\eqref{betas}.  The
sample value of the country-level transmission rate was then computed
by
\begin{equation}
  \label{eta}
  \eta = \frac{\bar{\beta} I(2015)}{
    \beta_A A(2015) + \beta_U \big(U(2015) + D(2015) + T(2015)\big)
    + \beta_V V(2015)},
\end{equation}
derived by rearranging eq.~\eqref{foi}, and uses the model initial
conditions eq.~\eqref{initial_conditions}.  This estimate of $\eta$ is
derived for the 2015 estimates of the proportion of PLHIV divided
among acute infections; undiagnosed, diagnosed, or treated infections;
and virally suppressed infections.  Using the estimate of $\eta$ in
projections for future times, when these proportions may change, may
introduce bias.

A few countries—e.g. Bulgaria, Timor-Leste, and Yemen—are currently
estimated by UNAIDS to have low HIV prevalence but relatively high
incidence, resulting in large estimated transmission rates, leading to
rapid projected growth of their HIV epidemics by 2035. The apparent
low prevalence and high incidence may, at least in part, be due to the
resolution in UNAIDS estimates (0.1\% for prevalence, 0.01\% for
annual per-capita incidence).  Results for such countries should be
considered in light of this limitation.


\section{Uncertainty analysis}
\label{uncertainty}

Projections were generated for each country through model simulations
using 1000 random samples from published parameter distributions
(\autoref{model_param}) and the country-specific estimated
distribution of the transmission rate (\autoref{model_fitting}). The
parameters were sampled using Latin hypercube
sampling\cite{blower1994}.  We also used the variance reduction
technique of running different interventions with the same parameter
samples, i.e.  one set of sample parameter values is drawn, status quo
and intervention scenarios are all run with those sample values and
compared, and this process is repeated for each of the 1000 sample
sets of parameters\cite{shechter2006}.

The model outcomes were summarized using the median, 1st and 3rd
quartiles (i.e. 25th and 75th percentiles), and 5th and 95th
percentiles.  We calculated partial rank correlation coefficients
(PRCC)\cite{blower1994} to measure the independent effect of each
non-vaccine parameter on model projections of global new HIV
infections from 2015 to 2035 (Extended Data Fig.~\ref*{PRCCs}).

Given that vaccine parameters have not yet been quantified, we
restricted the number of unknowns by keeping vaccine efficacy constant
over time.  This assumption is based on proposed vaccine regimens that
include boosters for sustaining immunogenicity.  We chose plausible
baseline values for the included vaccine parameters (efficacy 50\%,
ultimate coverage 70\%, speed of rollout 25\% per year, year vaccine
is first available 2020) and examined uncertainty by simulating 6
additional vaccine scenarios, varying efficacy (30\%, 70\%), ultimate
coverage (50\%, 90\%), speed of rollout (10\% coverage per year), and
first year of availability (2025; Fig.~4). The vaccine scenario
analysis was done using the modes from the parameter distributions,
not samples from these distributions, to reduce the computation time.

We used several probability distributions to quantify the parameters.
\begin{description}
\item[Uniform] $\Uniform(a, b)$ is the uniform random variable with
  minimum $a$ and maximum $b$, which has density function
  \begin{equation}
    \label{uniform}
    f_{\Uniform}(x) =
    \begin{cases}
      \frac{1}{b - a} & \text{if $a \leq x \leq b$,}
      \\
      0 & \text{otherwise.}
    \end{cases}
  \end{equation}
  We defined the mode for the uniform distribution to be the midpoint
  $\frac{b - a}{2}$.

\item [Triangular] $\Triangular(a, b, c)$ is the triangular random
  variable distribution with minimum $a$, mode $b$, and maximum $c$.
  It has density function
  \begin{equation}
    \label{triangular}
    f_{\Triangular}(x) =
    \begin{cases}
      \frac{2 (x - a)}{(c - a)(b - a)} & \text{if $a \leq x \leq b$,}
      \\
      \frac{2 (c - x)}{(c - a)(c - b)} & \text{if $b \leq x \leq c$,}
      \\
      0 & \text{otherwise.}
    \end{cases}
  \end{equation}

\item[Beta-PERT] $\BetaPERT(a, b, c)$ is the Beta-PERT probability
  distribution\cite{malcom1959} with minimum $a$, mode $b$, and
  maximum $c$.  It has density function
  \begin{equation}
    \label{BetaPERT}
    f_{\BetaPERT}(x) =
    \begin{cases}
      \frac{(x - a)^{v - 1} (c - x)^{w - 1}}{(c - a)^{v + w - 2} B(v, w)}
      & \text{if $a \leq x \leq c$,}
      \\
      0 & \text{otherwise,}
    \end{cases}
  \end{equation}
  where
  \begin{equation}
    \begin{split}
      \mu &= \frac{a + \lambda b + c}{\lambda + 2},
      \\
      v &= \frac{(\mu - a)(2 b - a - c)}{(b - \mu) (c - a)},
      \\
      w &= \frac{v (a - \mu)}{\mu - c},
    \end{split}
  \end{equation}
  and $B(v, w)$ is the standard beta function\cite{davis1972}.
  % \cite[\S6.2]{davis1972}
  We used the standard value of $\lambda = 4$.

\item[Lognormal] $\Lognormal(\mu, \sigma^2)$ is the lognormal random
  variable, which has density function
  \begin{equation}
    f_{\Lognormal}(x) = \frac{1}{x \sigma \sqrt{2 \pi}}
    \exp\left(- \frac{\left(\log x - \mu\right)^2}{2 \sigma^2}\right).
  \end{equation}
\end{description}


% \printbibliography
\bibliography{supplementary_information}
\bibliographystyle{naturemag}

\end{document}
