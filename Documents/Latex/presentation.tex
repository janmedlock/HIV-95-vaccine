%%%%%%%%%%%%%%%%%%%%%%%%%%%%%%%%%%%%%%%%%%%%%%%%%%%%%%%%%%%%
%%  This Beamer template was created by Cameron Bracken.
%%  Anyone can freely use or modify it for any purpose
%%  without attribution. with  modify it for any purpose
%%  without attribution without who freely modify
%%  Last Modified: January 9, 2009
%%

\documentclass[xcolor=x11names,compress]{beamer}

%% General document %%%%%%%%%%%%%%%%%%%%%%%%%%%%%%%%%%
\usepackage{graphicx}
\usepackage{tikz}
\usepackage{multirow}
\usepackage{hyperref}
\usepackage[round]{natbib}
\usepackage{hypernat}
\usetikzlibrary{decorations.fractals}
\usepackage{comment}

%\usepackage{movie15}
%%%%%%%%%%%%%%%%%%%%%%%%%%%%%%%%%%%%%%%%%%%%%%%%%%%%%%


%% Beamer Layout %%%%%%%%%%%%%%%%%%%%%%%%%%%%%%%%%%
\useoutertheme[subsection=false,shadow]{miniframes}
\setbeamertemplate{footline}{%
\begin{beamercolorbox}{subsection in head/foot}
    \color{DodgerBlue4}\vskip2pt~ \insertsubsection % footnote color
    \hfill \insertpagenumber{} %
    of \insertpresentationendpage{} ~\vskip2pt
\end{beamercolorbox}
}


\setbeamertemplate{blocks}[rounded, shadow=true]
\setbeamertemplate{navigation symbols}{}
\usepackage{amsmath}
\useinnertheme{default}
\usefonttheme{serif}
\usepackage{palatino}
\usepackage{fancybox}
% \setbeamerfont{title like}{shape=\scshape}
% \setbeamerfont{frametitle}{shape=\scshape}

% \setbeamercolor*{lower separation line head}{bg=DodgerBlue4} % header color
% \setbeamercolor*{normal text}{fg=black}
% \setbeamercolor*{alerted text}{fg=red}
% \setbeamercolor*{example text}{fg=black}
% \setbeamercolor*{structure}{fg=black}
% \setbeamercolor*{palette tertiary}{fg=black,bg=black!10}
% \setbeamercolor*{palette quaternary}{fg=black,bg=black!10}
% \setbeamercolor*{block title}{fg=white,bg=orange} % block color
% \setbeamercolor*{block body}{bg = white,fg=black}
% \setbeamercolor{bibliography item}{fg=⟨red⟩}
% \setbeamercolor*{bibliography entry title}{fg = red}
% \setbeamercolor*{bibliography entry journal}{fg=red}
% \setbeamercolor*{bibliography entry author}{fg = red}

%\usetheme{PaloAlto}
\setbeamertemplate{headline}{}


\newcommand{\md}{\mathrm{d}}
\newcommand{\mD}{\mathrm{D}}
\renewcommand{\(}{\begin{columns}}
\renewcommand{\)}{\end{columns}}
\newcommand{\<}[1]{\begin{column}{#1}}
\renewcommand{\>}{\end{column}}
%%%%%%%%%%%%%%%%%%%%%%%%%%%%%%%%%%%%%%%%%%%%%%%%%%
\usebackgroundtemplate{
 \tikz[overlay,remember picture]
   \node[at=(current page.south west),anchor=south west,inner  sep=13pt]{
     \includegraphics[scale=0.10]{CIDMA}};}



\begin{document}

%%%%%%%%%%%%%%%%%%%%%%%%%%%%%%%%%%%%%%%%%%%%%%%%%%%%%%
%%%%%%%%%%%%%%%%%%%%%%%%%%%%%%%%%%%%%%%%%%%%%%%%%%%%%%
%\section{\scshape Introduction}
% \begin{frame}

% \title{Cost-effectiveness of WHO's 90-90-90 goal for HIV}
% %\author{Abhishek Pandey, Dan Yamin, Jeffrey Townsend \& Alison Galvani}
% \date{\textit{Center for Infectious Disease Modeling and Analysis} \\
% \today}
%  \institute[CIDMA]
%  {
%    \includegraphics[scale=0.15]{CIDMA}
%  }


% % \date{
% % 	\begin{tikzpicture}[decoration=Koch curve type 2]
% % 		\draw[DeepSkyBlue4] decorate{ decorate{ decorate{ (0,0) -- (3,0) }}};
% % 	\end{tikzpicture} \\
% % 	\vspace{1cm}
% % 	\today
% % }
% \titlepage
% \end{frame}

% \begin{frame}{Research Problem:}

% {{\small\begin{minipage}{1 \textwidth}
% \textcolor{red}{90-90-90 goal by 2020:}
% \begin{itemize}
% \item 90\% of all people living with HIV will know their HIV status (diagnosed)
%  \vskip 0.05in
% \item 90\% of all people with diagnosed HIV infection will receive sustained antiretroviral therapy  \vskip 0.05in
% \item 90\% of all people receiving antiretroviral therapy will have viral suppression.
% \end{itemize}
% \end{minipage}}}
% \vskip 0.1in
% {{\small\begin{minipage}{1 \textwidth}
% \textcolor{red}{Research Questions:}
% \begin{itemize}
% \item What is the effectiveness of reaching WHO goal for each country?
% \item For what countries, achieving the WHO goal in
%   the next 5 years is cost effective.  \vskip 0.05in
% \item For countries for which the goal is not cost effective,
%   what combined strategy is cost-effective?  \vskip 0.05in
% \end{itemize}
% \end{minipage}}}

% \end{frame}

% \begin{frame}{Disability-Adjusted Life Year (DALY)}


% {{\small\begin{minipage}{1 \textwidth}
% \textcolor{red}{Definition:} \\
% The disability-adjusted life year (DALY) is a measure of overall
% disease burden, expressed as the number of years lost due to
% ill-health, disability or early death. \vskip 0.2in
% \textcolor{red}{Calculation:} \\
% DALY = Year of life lost due to disability + Year of life lost due to
% premature mortality.  \vskip 0.2in
% \textcolor{red}{Effectiveness:} \\
% We measured effectiveness of achieving WHO goal of 90-90-90 in terms
% of number of DALYS saved.
% \end{minipage}}}

% \end{frame}


% \begin{frame}{Cost-Effectiveness}

% {{\small\begin{minipage}{1 \textwidth}
% \textcolor{red}{Program cost:} \\
% Program cost is the incremental cost required to meet the 90-90-90 goal.
%  \vskip 0.2in
% \textcolor{red}{Cost effectiveness} \\
% An intervention that per DALY avoided, costs:
% \begin{itemize}
% \item  less than 3 times the annual GDP per capita is Cost-effective,
% \item  less than the annual GDP per capita is very cost-effective,
% \item less than it costs to continue status quo is cost-saving
%   (Not sure about this def).
% \item  more than 3 times the annual GDP per capita is not
%   cost-effective
% \end{itemize}
% \vskip 0.2in
% \textcolor{red}{Incremental cost effectiveness ratio (ICER):}\\
% is the difference in cost between achieving 90-90-90 and continuing status
% quo , divided by the difference in DALYS averted.

% \end{minipage}}}


% \end{frame}

\begin{frame}{Status Quo}
\vspace{-0.4in}
\hspace{-0.4cm}
\begin{centering}
%\includefullwidthgx{../../initial_proportions}
 \includegraphics[width= \textwidth, height=0.62\textheight]{../../initial_proportions}
\end{centering}
\end{frame}



\begin{frame}{Effectiveness: Prevalence}
\href{https://drive.google.com/open?id=0B_53qFSHU3XKUDVfUWVtd1BPc1E}{
\begin{centering}
\includegraphics[width=\textwidth]{../../prevalence}
\end{centering}
}

\end{frame}

\begin{frame}{Effectiveness: New infections averted}


\href{https://drive.google.com/open?id=0B_53qFSHU3XKYzhlQjNzRUdRVXc}{
\begin{centering}
\includegraphics[width=\textwidth]{../../infections_averted}
\end{centering}}
\end{frame}


\begin{frame}{Effectiveness: DALYs}
\begin{centering}
\includegraphics[width=\textwidth]{../../909090effectiveness}
\end{centering}


\end{frame}

\begin{frame}{Cost-Effectiveness}
\begin{centering}
\includegraphics[width=\textwidth]{../../909090ICER}
\end{centering}
\end{frame}

% \begin{frame}{Summary}
% \begin{table}
% \begin{center}
% \begin{tabular}{l|r|r|r|r|r}
% \hline
% Country  & USA & Nigeria & Rwanda & India & Burundi  \\
% \hline
% DALYs averted &  &  & & & \\
% Incremental cost &  &  & & & \\
% ICER &  &  & & & \\
% Per capita GDP &  &  & & & \\
% \hline
% \end{tabular}
% \end{center}
% \end{table}
% \end{frame}

% \begin{frame}{Future Work}


% {{\small\begin{minipage}{1 \textwidth}
% \begin{itemize}
% \item What other combined strategy may be cost-effective if not
%   90-90-90 for a country.
% \item And in general, is there any optimal three part target for each
%   country other than 90-90-90?
% \item Expanding the results to more countries.
% \end{itemize}
% \end{minipage}}}


% \end{frame}




\end{document}
